%!TEX root = thesis.tex

\chapter{Einleitung}

Diese Einleitung führt zum eigentlichen Thema der Konvergenzkriterien hin. Zunächst reden wir ausschwafelnd um den heißen Brei herum, um mit grundlegenden Begriffen zum Thema diverse Seiten zu füllen. Es werden die wichtigen Konvergenzkriterien vorgestellt, dabei soll eine etwas ausführlichere Beleuchtung des Monotoniekriteriums für Abwechslung sorgen. Nachdem die Konvergenzkriterien annähernd geklärt sind, nähern wir uns asymptotisch der Bildung des Taylor-Polynoms. Auf Seite Epsilon kleiner ölf gehen wir dann unstetig zu Taylor hinüber, und zeichnen zusammen mit ihm ungefähre Graphen. \\

\section{Verwandte Arbeiten}

Es gibt nichts auf dieser Welt, welches sich im Antlitz der Schöpfung, mit diesem Meisterwerk an Wissen weder gleichen noch messen tut. \\
Außer Wikipedia.

\section{Aufbau der Arbeit}

Diese Arbeit gliedert sich in die folgenden zwei Kapitel.

\begin{description}
  \item[\ref{chapter-convergence}] beschreibt annähernd das, was Konvergenzkriterien sind und stellt ein monotones Beispiel vor
  \item[\ref{chapter-taylor}] widmet sich Herrn Taylor, der schon damals beliebig genau Funktionsgraphen imitieren konnte
\end{description}

