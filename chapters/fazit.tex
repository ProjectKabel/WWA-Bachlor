%!TEX root = thesis.tex

\chapter{Zusammenfassung und Ausblick}
\label{chapter-fazit}

Die Konvergenzkriterien haben gezeigt, dass Grenzwerte im Wesentlichen erst nach unendlich vielen Schritten erreicht werden.
Es ist hoffnungslos. Man könnte hier noch hunderte Seiten beschreiben, aber das Aufwand-/Nutzen konvergiert mit jedem neuen Blatt weiter gegen Null. Auf dem Gebiet der Konvergenz ist noch viel Arbeit nötig, einen guten Beitrag dazu haben zum Beispiel Gottfried Wilhelm Leibniz \cite{cauchy1682} in 1682 und  Augustin Louis Cauchy in 1821 bereits geleistet.
Dennoch ist festzustellen, dass das Supremum vermutlich nie erreicht wird.

Taylorpolynome bieten eine effiziente Art sich bestimmten Intervallen einer Funktion anzunähern. 